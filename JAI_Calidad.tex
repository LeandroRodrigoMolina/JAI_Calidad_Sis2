\documentclass[a4paper,10pt]{article}

% Paquetes requeridos
\usepackage[utf8]{inputenc}
\usepackage[spanish]{babel}
\usepackage{csquotes}
\usepackage{amsmath, amssymb, amsfonts}
\usepackage{graphicx}

% Añadir al inicio del documento después de \usepackage{graphicx}
\usepackage{tikz}
\usepackage[style=apa, backend=biber, natbib=true, language=spanish, url=true]{biblatex}
\usepackage{tocloft} % Para personalizar el índice
\usepackage[left=3.5cm,right=2.5cm,top=3.5cm,bottom=3.8cm]{geometry}
\usepackage{setspace} % Espaciado
\usepackage{titlesec} % Para personalizar los títulos
\usepackage{fancyhdr} % Para personalizar encabezados y pies de página
\usepackage{newtxtext}
\usepackage{ragged2e}
\usepackage{caption}
\usepackage{footnote}

\makesavenoteenv{figure}
% Configuración para las etiquetas de figura
\captionsetup[figure]{labelformat=empty} % Esto elimina el número de figura y el separador

\pagestyle{fancy}
\fancyhf{} % Limpia encabezados y pies de página
\renewcommand{\headrulewidth}{0pt} % Elimina la línea del encabezado

\addbibresource{referencias.bib}
\DeclareLanguageMapping{spanish}{spanish-apa}
% Configuraciones
\setlength{\parskip}{6pt} % Espacio entre párrafos
\setstretch{1.15} % Espacio entre líneas

\renewcommand{\cftsecleader}{\cftdotfill{\cftdotsep}} % Para puntos en el índice

% Estilos para títulos y subtítulos
\titleformat{\section}
{\normalfont\fontsize{12}{15}\bfseries}{\thesection}{1em}{}
\titleformat{\subsection}
{\normalfont\fontsize{10}{13}\bfseries}{\thesubsection}{1em}{}
\titleformat{\subsubsection}
{\normalfont\fontsize{10}{13}\bfseries}{\thesubsubsection}{1em}{}

\usepackage[hypertexnames=false, colorlinks=true, 
linkcolor=blue, 
citecolor=blue, 
urlcolor=blue, 
linkbordercolor={1 1 0}, 
citebordercolor={1 1 0}, 
urlbordercolor={1 1 0}, 
filecolor=blue, 
pdfborderstyle={/S/U/W 1}]{hyperref}

% Inicio del documento
\begin{document}
	% Carátula
	\centering
	{\fontsize{14}{17}\bfseries Sistema de gestión empresarial. Aseguramiento de la calidad de proyectos de software.\par}
	{\small Borgo, Martín Alejandro; Molina, Leandro Rodrigo; Confalonieri, Juan; Tenich, Javier; Laner Sandoval, Anacleto José Federico; Panozzo Jeremias.\par}
	{\normalsize Universidad Nacional de Entre Ríos\par}
	{\normalsize Facultad de Ciencias de la Administración\par}
	{\normalsize Licenciatura en Sistemas\par}
	{\small
		\href{mailto:martinborgo8@gmail.com}{martinborgo8@gmail.com},
		\href{mailto:LeandroRodrigoMolina@gmail.com}{LeandroRodrigoMolina@gmail.com}
		\href{mailto:juanconfaa@gmail.com}{juanconfaa@gmail.com}
		\href{mailto:jtenich@gmail.com}{jtenich@gmail.com}
		\href{mailto:josecitolansan@yahoo.com.ar}{josecitolansan@yahoo.com.ar}
		\href{mailto:jeremiaspanozzo@gmail.com}{jeremiaspanozzo@gmail.com}
		\par}	
	% Resumen y palabras clave
	{\begin{quote} \small \justify\textbf{Abstract.} A medida que el software se integra en la vida cotidiana de las personas, la preocupación por la calidad del mismo se hace mayor, preocupando a los programadores, diseñadores, testers y demás interesados en el desarrollo del mismo. Este artículo tiene como objetivo desarrollar un plan de aseguramiento de la calidad para el software "Sistema de Gestión Empresarial" (SGE). Basado en el estándar IEEE 730, el plan se encuentra adaptado para un prototipo incompleto, y busca garantizar que el software cumpla con los requerimientos técnicos y de rendimiento especificado. Se abordan aspectos como la evaluación de requerimientos, diseño, acciones correctivas, pruebas y gestión de riesgos del software. El artículo concluye que un plan de calidad bien definido es crucial para alcanzar los objetivos de calidad en proyectos de software de mediana y gran envergadura. \end{quote} \par}
	{\begin{quote} \small \justify\textbf{Keywords:} Calidad de Software, Aseguramiento de Calidad, Software.\end{quote} \par}
	
	\justifying
	
	\section{Introducción}
	En sectores más convencionales como la industria de manufacturas, el concepto de calidad fue establecido por los distintos participantes del mercado, en parte por la existencia de un bien físico que puede ser inspeccionado no solo a través de los sentidos sino también a partir de ciertos estándares propios de cada fabricante.
	
	Aplicar este concepto a la industria del software resulta difícil en la mayoría de los casos, en parte por la inexistencia de un producto físico y más importante aún, es el hecho de que cuando hablamos de software, la palabra calidad puede tener diversas connotaciones, la cual varía de acuerdo al rol y posición del individuo que analiza el programa en cuestión. Por ejemplo, para los usuarios finales, la “calidad de software” radica en el funcionamiento libre de defectos, la fiabilidad y facilidad de uso del sistema, un nivel aceptable de tolerancia al fallo, entre otras. Para el desarrollador, la calidad recae en el cumplimiento de las especificaciones, definida tanto por la organización en cuestión u otros entes asociados con la industria y estándares. Para los organismos de estándares, la calidad se centra en proteger la reputación de la industria, prevenir el fraude, evitar demandas legales y abordar las preocupaciones de los consumidores, entre otros aspectos.
	
	Debido a que existen diferentes connotaciones de la palabra “calidad”, es necesario dar una definición que contemple cada uno de estos puntos de vista. Según la ISO 9000 la calidad se define como “el grado en que un conjunto de características inherentes cumple con los requerimientos” \parencite{ISO_9001_2015}. Si bien esta norma puede ser aplicada en la construcción de software, autores como Pressman \parencite{Pressman_2010} define a la calidad como la “conformidad con los requerimientos funcionales y de rendimiento explícitamente establecidos, con los estándares de desarrollo documentados, y con las características implícitas que se esperan de todo software desarrollado profesionalmente”. Esta definición sugiere tres requerimientos para garantizar de calidad que deben ser cumplidos por el desarrollador:
	
	\begin{itemize}
		\item Requerimientos funcionales específicos, refieren principalmente a las salidas del sistema de software.
		\item Estándares de calidad del software mencionados en el contrato.
		\item Buenas prácticas de ingeniería de software, reflejan prácticas profesionales de vanguardia, estas deben ser cumplidas por el desarrollador, incluso si no se mencionan explícitamente en el contrato.
	\end{itemize}
	
	Según W. Edwards Deming \parencite{Deming_2018}, la aplicación de un buen proceso de desarrollo de software produce un software de calidad. En este contexto, un plan de calidad es un documento que especifica qué procedimientos y recursos deberían aplicarse, quién debe aplicarlos y cuándo aplicarlos a un proyecto, proceso, producto o contrato específico, de manera de poder alcanzar los objetivos de la calidad \parencite{Álvarez_López_2005}, es decir, un documento que garantiza la calidad del software elaborado, en base a las expectativas del usuario y a los estándares utilizados.
	
	El aseguramiento de la calidad (QA) es el proceso encargado de verificar que se estén aplicando correctamente los procesos y estándares previamente definidos en el plan de calidad, los cuales tiene como objetivo asegurar la calidad del producto final \parencite{Sommerville_2011}.
	
	Este trabajo se llevó a cabo en el marco de la cátedra de Metodología de Sistemas II, con el objetivo de aplicar los conocimientos adquiridos en clase mediante el desarrollo de un plan de calidad para el software "Sistema de Gestión Empresarial" (SGE). Este plan será implementado en el software mencionado y posteriormente revisado por un grupo externo.
	
	\section{Desarrollo de trabajo}
	El desarrollo del documento de calidad se realizará siguiendo el estándar IEEE 730 \parencite{IEEE_Standards_Association} para la especificación de los requerimientos y directrices del plan de calidad de software. En este trabajo, se adaptará el estándar debido a que la implementación del plan de calidad elaborado se realizará sobre un prototipo incompleto, en línea con el enfoque didáctico y educativo del presente trabajo. En las siguientes secciones se elaborarán las partes del plan de calidad, el cual consta de la siguiente estructura.
	
	\begin{itemize}
		\item Propósito.
		\item Organización.
		\begin{itemize}
			\item Evaluación de Requerimientos.
			\item Evaluación de diseño de Software.
			\item Evaluación del Proceso de Acción Correctiva.
			\item Revisión y Auditoría.
		\end{itemize}
		\item Documentación.
		\item Pruebas.
		\item Informe de Problemas y Acción Correctiva.
		\item Gestión de Riesgo.
		\begin{itemize}
			\item Identificación de Riesgos.
		\end{itemize}
	\end{itemize}
	
	\subsection{Propósito}
	Se define como propósito al Plan de Aseguramiento de la Calidad del proyecto para el SGE (Sistema de Gestión Empresarial), estableciendo la organización, tareas y responsabilidades del equipo de aseguramiento de calidad. Además, se proporcionan guías, herramientas, técnicas y metodologías para la ejecución de actividades y la elaboración de los reportes de calidad.
	
	El objetivo del plan es verificar que el software y la documentación a ser entregada cumplan con todos los requerimientos técnicos. Los procedimientos definidos en este documento se utilizan para examinar las prestaciones que dará el software, así como para examinar la documentación y determinar que ambos cumplieron con los requerimientos técnicos y de rendimiento del sistema a ser desarrollado.
	
	La siguiente lista identifica los elementos del software que abarcará este plan:
	\begin{itemize}
		\item Interfaz de Usuario (UI).
		\item Modelo de Datos.
	\end{itemize}
	
	\subsection{Organización}
	En esta sección se muestran todas las tareas a cargo del equipo de calidad, estas tareas se realizan a lo largo del ciclo de vida del proyecto, y cumpliendo los plazos establecidos en el plan de desarrollo de software.
	
	Una tarea se considera completa si se ha elaborado el reporte correspondiente. Adicionalmente, las siguientes tareas que se describirán requieren la coordinación y cooperación del equipo de desarrollo para llevarlas a cabo satisfactoriamente por el equipo de calidad.
	
	\subsubsection{Evaluación de Requerimientos}
	El análisis de requerimientos establece un acuerdo formal entre el equipo encargado del desarrollo del proyecto y los clientes o partes interesadas en el mismo. Se debe mantener y establecer un acuerdo con los usuarios finales del sistema con el propósito de realizar correctamente el análisis de requerimientos del mismo.
	
	Las actividades a cargo del equipo de calidad abarcan las siguientes tareas:
	\begin{itemize}
		\item Revisar los requerimientos para determinar su claridad y correctitud.
		\item Verificar que los cambios en el documento de requerimientos del sistema sean seguidos, revisados y comunicados al equipo de desarrollo.
		\item Verificar que los compromisos con los clientes sean apropiadamente documentados y comunicados al equipo de desarrollo.
		\item Verificar los procesos descritos con el fin de definir, documentar y localizar requerimientos aún no documentados.
		\item Verificar que los requerimientos estén apropiadamente documentados, controlados y seguidos.
	\end{itemize}
	
	El formato de la documentación será acordado previamente con el administrador del proyecto.
	
	\subsubsection{Evaluación del Diseño del Software}
	El objetivo del proceso de diseño es tomar decisiones sobre la estructura de los componentes que utilizará el sistema como así también de la arquitectura bajo la cual se desarrollará este último. El nivel de detalle en el diseño debe ser tal que el código de los distintos módulos pueda ser implementado por personas que no formaron parte del equipo de diseño.
	
	Las actividades a cargo del asegurador de calidad son las siguientes:
	\begin{itemize}
		\item Verificar que el proceso de diseño del software siga los estándares acordados por el equipo de diseño.
		\item Verificar que todos los requerimientos se vean plasmados en el diseño.
		\item Revisar y auditar el contenido de los documentos de diseño del sistema.
		\item Determinar las acciones correctivas para aquellos casos en donde surja un incumplimiento de los estándares establecidos.
	\end{itemize}
	
	\subsubsection{Evaluación del Proceso de Acciones Correctivas}
	El proceso de acción correctiva implica identificar el problema y la corrección realizada durante el desarrollo del software, reportar el problema a la autoridad correspondiente, analizar y realizar una corrección oportuna y competente para posteriormente registrar y dar seguimiento a cada problema.
	
	Las actividades que debe realizar el equipo de calidad son las siguientes:
	\begin{itemize}
		\item Revisar el proceso de acción correctiva y sus resultados.
		\item Documentar los resultados de estas tareas correctivas de acuerdo al formato establecido previamente. Este documento debe ser entregado al administrador del proyecto.
		\item Coordinar cada acción correctiva con el administrador del proyecto.
	\end{itemize}
	
	\subsubsection{Revisión y Auditoría}
	El equipo de calidad verifica de forma periódica el estado del proyecto, el progreso y los problemas del mismo, proporcionando de esta manera la siguiente información a la dirección del proyecto:
	\begin{itemize}
		\item El nivel de cumplimiento del proyecto de acuerdo a los plazos y funcionalidades definidas en los diferentes documentos que envuelven al proyecto.
		\item Identificación de los problemas existentes o potenciales en las distintas áreas y equipos de desarrollo.
	\end{itemize}
	
	Debido a la índole crítica de la tarea, el equipo de calidad puede comunicar libremente los resultados obtenidos a la dirección del proyecto y al equipo de desarrollo.
	
	\subsection{Documentación}
	La documentación que describe el SGE y el proceso de desarrollo de software se crea y actualiza periódicamente a lo largo de todo el ciclo de desarrollo. Los documentos listados deben estar bajo la administración de la configuración, y se envía una petición al Administrador de la Configuración de Software (ACS) cuando se realizan cambios, para que este determine si el documento puede entrar en la versión de control.
	
	\begin{itemize}
		\item Especificación de requerimientos de Software: Describe los requerimientos funcionales y no funcionales del software SGE.
		\item Plan de Aseguramiento de Calidad: Describe los planes y roles que debe adoptar cada uno de los interesados en el desarrollo del software SGE.
		\item Plan de Pruebas: Describe la nomenclatura utilizada en el proyecto así como la forma en que se determina la línea base.
		\item Plan de desarrollo de software: Describe lo que se va a implementar, los calendarios, actividades y responsabilidades de los miembros del equipo de desarrollo.
	\end{itemize}
	
	Todos los documentos mencionados están sujetos a los siguientes criterios de revisión aplicados a lo largo del ciclo de vida del proyecto:
	\begin{itemize}
		\item Claridad y Precisión: El contenido debe ser claro y comprensible para todos los miembros del equipo, evitando ambigüedades.
		\item Consistencia: La información debe ser coherente y uniforme entre todos los documentos.
		\item Actualización y Control de Cambios: Los documentos deben estar sujetos a control de versiones y cambios, asegurando que cualquier modificación sea aprobada por el ACS antes de ser implementada.
		\item Cumplimiento de Estándares: Confirmar que cada documento cumple con los estándares establecidos (por ejemplo, IEEE 730).
	\end{itemize}
	
	\subsection{Pruebas}
	Las actividades de pruebas que se realizan para el proyecto SGE son los siguientes:
	\begin{itemize}
		\item Pruebas de Cajas Negras.
		\item Pruebas de Aceptación.
		\item Validación del Modelo de Datos.
	\end{itemize}
	
	El administrador del proyecto debe asignar una persona como líder del equipo de pruebas. El líder de equipo es el responsable de elaborar un plan de pruebas para el software SGE. El personal que haya desarrollado algún caso de uso para el proyecto realiza las siguientes actividades de pruebas necesarias para el software:
	\begin{itemize}
		\item Realizar las pruebas de interfaz.
		\begin{itemize}
			\item Verificar la usabilidad.
			\item Verificar las diferentes entradas de los formularios.
			\item Verificar el correcto funcionamiento y uso de las ventanas y/o alertas emergentes.
			\item Asegurar la navegabilidad.
			\item Asegurar que sea intuitiva para el usuario.
		\end{itemize}
		\item Verificar los modelos de datos que utiliza el sistema.
	\end{itemize}
	
	El equipo de calidad está a cargo de auditar las actividades antes descritas y verificar que la documentación de pruebas sea adecuada, completa, correcta y aprobada antes de su utilización.
	
	\subsection{Informe de Problemas y Acción Correctiva}
	El propósito de esta sección es describir el reporte y control del sistema utilizado por el equipo de calidad para registrar y analizar las discrepancias encontradas, así como monitorear la implementación de las acciones correctivas.
	
	El equipo de calidad reporta el resultado de las auditorías y las recomendaciones proporcionadas, con el fin de asegurar que el proceso se esté siguiendo de manera correcta y se está trabajando de forma efectiva. El proceso de reporte de auditorías está dirigido hacia el administrador del proyecto, el cual utiliza los reportes para:
	\begin{itemize}
		\item Saber si los procesos de desarrollo son acatados y efectivos para el cumplimiento de las metas del proyecto. El administrador de proyecto puede iniciar un cambio en el proceso cuando lo requiera siempre que se siga las medidas y procedimientos establecidos, con el fin de asegurar la estabilidad del proceso.
		\item Para indicar el acuerdo, desacuerdo, o el aplazamiento de las recomendaciones realizadas durante el proceso del reporte de auditoría. En caso que el administrador del proyecto indique el desacuerdo con las recomendaciones registradas en el reporte de auditoría, la disposición final de recomendaciones del informe se hace por los mecenas del proyecto.
	\end{itemize}
	
	Al encontrarse un problema que impacta en la calidad del proyecto, ya sea en algún documento, código o producto relacionado se tendrá que contactar con el responsable del componente. En casos de desacuerdos con el equipo de calidad se debe notificar al administrador del proyecto para que intervenga y proporcione una solución al problema. En caso de que el administrador del proyecto no haya dado una solución, el problema se elevará a los patrocinadores del proyecto para la toma final de una decisión.
	
	\subsection{Gestión de Riesgo}
	El propósito de la gestión de riesgo es identificar y mitigar los posibles riesgos que puedan afectar el desarrollo, implementación y funcionamiento del software SGE. Este proceso permite reducir la probabilidad de que estos riesgos ocurran y minimizar su impacto si llegan a materializarse.
	
	\subsubsection{Identificación de Riesgos}
	\begin{itemize}
		\item Riesgos Técnicos: Falta de experiencia en tecnologías, fallos en la integración de componentes, o problemas de desempeño.
		\item Riesgos de Planificación: Retrasos en el cronograma debido a dependencias externas o problemas con la asignación de recursos.
		\item Riesgos de Calidad: Defectos no detectados en fases tempranas, que afecten la funcionalidad del sistema.
		\item Riesgos de Requerimientos: Cambios inesperados en los requerimientos del cliente o falta de claridad en los mismos.
	\end{itemize}
	
	Cada riesgo se evalúa según su probabilidad de ocurrencia (baja, media, alta) y su impacto (bajo, medio, alto). Esta evaluación se utiliza para priorizar los riesgos y centrar los esfuerzos de mitigación en los más críticos. Además, por cada riesgo identificado, se elabora un plan de mitigación que incluye acciones preventivas y de contingencia.
	
	\begin{itemize}
		\item Acciones preventivas: Reducir la probabilidad de que ocurra el riesgo.
		\item Acciones de contingencia: Tener planes alternativos listos si el riesgo se materializa.
	\end{itemize}
	
	Los reportes tienen que ser reportados regularmente al director del proyecto, estos reportes deben poseer la siguiente información:
	\begin{itemize}
		\item Los riesgos detectados y su ocurrencia.
		\item El estado actual de los riesgos y las acciones de contención.
		\item Los posibles nuevos riesgos que puedan surgir.
		\item Acciones de prevención y de contingencia propuestos.
	\end{itemize}
	
	\section{Resultados obtenidos}
	En el documento desarrollado en la sección anterior se especificó el plan de calidad, dentro del cual se definen los criterios de calidad que cumplen con las expectativas del cliente, las pruebas de funcionalidad como de usabilidad y aceptación. No obstante, varias secciones de la estructura original de la IEEE 730 fueron omitidas debido al objetivo final del trabajo.
	
	El plan de aseguramiento de calidad resulta útil durante el proceso de construcción de software, puesto que facilita y agiliza la comunicación y coordinación entre los miembros del equipo. Además, brinda una mayor claridad sobre los objetivos de calidad específicos del proyecto, junto con los métodos que se deben utilizar para alcanzar estos mismos, asegurando de esta forma el correcto cumplimiento de los requerimientos y por ende la satisfacción del cliente y partes interesadas. Finalmente, cumple con los estándares de calidad con los cuales el equipo de calidad se ha comprometido a cumplir.
	
	\section{Conclusiones}
	Si bien el concepto de calidad varía de acuerdo al punto de vista del observador. Para el desarrollo de software la conformidad con los requerimientos funcionales y de rendimiento explícitamente establecidos, en conjunto con los estándares de desarrollo de documentos y las características implícitas que se esperan de todo software desarrollado profesionalmente es una de las definiciones más aceptadas. Por esta razón, para medianos y grandes proyectos de software, es de suma importancia definir un plan de calidad que documente los procedimientos y recursos que deberían aplicarse, quién debe aplicarlos y cuándo aplicarlos a un proyecto, proceso, producto o contrato específico, de manera de poder alcanzar los objetivos de la calidad.
	
	La implementación de un estándar para documentar la calidad, provee ciertas ventajas, como la uniformización de los documentos, brindando una serie de lineamientos para la correcta construcción de software lo que mejora la comunicación con el equipo de desarrollo. No obstante, en ocasiones la utilización y elaboración de este tipo de documentos resulta tediosa y demandante, ya que de acuerdo al tipo de proyecto se suele recomendar obviar ciertas secciones definidas en el estándar debido a que no suelen generar un impacto tan relevante en el desarrollo y los objetivos de calidad deseado para el proyecto.
	
	\nocite{*}
	\section{Referencias}
	\printbibliography[heading=none]
	\thispagestyle{empty} % No mostrar encabezado ni pie de página en esta página
	\begin{tikzpicture}[remember picture,overlay]
		\node[anchor=south east,opacity=0.12] at (current page.south east) {\includegraphics[width=3cm]{Kyria_Doro.png}};
	\end{tikzpicture}
\end{document}