\documentclass[a4paper,10pt]{article}

% Paquetes requeridos
\usepackage[utf8]{inputenc}
\usepackage[spanish]{babel}
\usepackage{csquotes}
\usepackage{amsmath, amssymb, amsfonts}
\usepackage{graphicx}
\usepackage[style=apa, backend=biber, natbib=true, language=spanish, url=true]{biblatex}
\usepackage{tocloft} % Para personalizar el índice
\usepackage[left=3.5cm,right=2.5cm,top=3.5cm,bottom=3.8cm]{geometry}
\usepackage{setspace} % Espaciado
\usepackage{titlesec} % Para personalizar los títulos
\usepackage{fancyhdr} % Para personalizar encabezados y pies de página
\usepackage{newtxtext}
\usepackage{ragged2e}
\usepackage{caption}
\usepackage{footnote}

\makesavenoteenv{figure}
% Configuración para las etiquetas de figura
\captionsetup[figure]{labelformat=empty} % Esto elimina el número de figura y el separador

\pagestyle{fancy}
\fancyhf{} % Limpia encabezados y pies de página
\renewcommand{\headrulewidth}{0pt} % Elimina la línea del encabezado

\addbibresource{referencias.bib}
\DeclareLanguageMapping{spanish}{spanish-apa}
% Configuraciones
\setlength{\parskip}{6pt} % Espacio entre párrafos
\setstretch{1.15} % Espacio entre líneas

\renewcommand{\cftsecleader}{\cftdotfill{\cftdotsep}} % Para puntos en el índice

% Estilos para títulos y subtítulos
\titleformat{\section}
{\normalfont\fontsize{12}{15}\bfseries}{\thesection}{1em}{}
\titleformat{\subsection}
{\normalfont\fontsize{10}{13}\bfseries}{\thesubsection}{1em}{}
\titleformat{\subsubsection}
{\normalfont\fontsize{10.5}{13}\bfseries}{\thesubsubsection}{1em}{}

\usepackage[hypertexnames=false, colorlinks=true, 
linkcolor=blue, 
citecolor=blue, 
urlcolor=blue, 
linkbordercolor={1 1 0}, 
citebordercolor={1 1 0}, 
urlbordercolor={1 1 0}, 
filecolor=blue, 
pdfborderstyle={/S/U/W 1}]{hyperref}

% Inicio del documento
\begin{document}
	% Carátula
	\centering
	{\fontsize{14}{17}\bfseries Plan de calidad SGC.\par}
	{\small Borgo, Martín Alejandro; Molina, Leandro Rodrigo; Confalonieri, Juan; Tenich, Javier; Sandoval, Jose; Panozzo Jeremias.\par}
	{\normalsize Universidad Nacional de Entre Ríos\par}
	{\normalsize Facultad de Ciencias de la Administración\par}
	{\normalsize Licenciatura en Sistemas\par}
	{\small
		\href{mailto:martinborgo8@gmail.com}{martinborgo8@gmail.com},
		\href{mailto:LeandroRodrigoMolina@gmail.com}{LeandroRodrigoMolina@gmail.com}
		\href{mailto:juanconfaa@gmail.com}{juanconfaa@gmail.com}
		\href{mailto:jtenich@gmail.com}{jtenich@gmail.com}
		\href{mailto:josecitolansan@yahoo.com.ar}{josecitolansan@yahoo.com.ar}
		\href{mailto:jeremiaspanozzo@gmail.com}{jeremiaspanozzo@gmail.com}
		\par}	
	% Resumen y palabras clave
	{\begin{quote} \small \justify\textbf{Abstract.} Aca abstract \end{quote} \par}
	{\begin{quote} \small \justify\textbf{Keywords:} Calidad de Software, Aseguramiento de Calidad, Requerimientos. \end{quote} \par}
	
	\justifying
	
	\section{Introducción}
	En sectores más convencionales como la industria de manufacturas, el concepto de calidad fue
	establecido por los distintos participantes del mercado, en parte por la existencia de un bien
	físico que puede ser inspeccionado no solo a través de los sentidos sino también a partir de
	ciertos estándares propios de cada fabricante.
	
	Aplicar este concepto a la industria del software resulta difícil en la mayoría de los casos, en
	parte por la inexistencia de un producto físico y más importante aún, es el hecho de que cuando
	hablamos de software, la palabra calidad puede tener diversas connotaciones, la cual varía de
	acuerdo al rol y posición del individuo que analiza el programa en cuestión. Por ejemplo para los
	usuarios finales, la “calidad de software” radica en el funcionamiento libre de defectos, la
	fiabilidad y facilidad de uso del sistema, un nivel aceptable de tolerancia al fallo, entre otras. Para
	el desarrollador, la calidad recae en el cumplimiento de las especificaciones, definida tanto por la
	organización en cuestión u otros entes asociados con la industria y estándares. Para los
	organismos de estándares, la calidad se centra en proteger la reputación de la industria,
	prevenir el fraude, evitar demandas legales y abordar las preocupaciones de los consumidores,
	entre otros aspectos.
	
	Debido a que existen diferentes connotaciones de la palabra “calidad”, es necesario dar una
	definición que contemple cada uno de estos puntos de vista. Según la ISO 9000 la calidad se
	define como "\textit{el grado en que un conjunto de características inherentes cumple con los requerimientos}" \parencite{ISO_9001_2015}. Si bien esta norma puede ser
	aplicada en la construcción de software, autores como \parencite{Pressman_2010} define a la calidad como
	la "\textit{conformidad con los requerimientos funcionales y de rendimiento explícitamente establecidos, con
		los estándares de desarrollo documentados, y con las características implícitas que se esperan de
		todo software desarrollado profesionalmente}." Esta definición sugiere tres requerimientos para la
		garantía de calidad que deben ser cumplidos por el desarrollador:
		\begin{itemize}
			\item Requerimientos funcionales específicos, refieren principalmente a las salidas del sistema de software.
			\item Estándares de calidad del software mencionados en el contrato.
			\item Buenas prácticas de ingeniería de software, reflejan prácticas profesionales de
			vanguardia, estas deben ser cumplidas por el desarrollador, incluso si no se mencionan
			explícitamente en el contrato.
		\end{itemize}
		Según \parencite{Deming_2018} la aplicación de un buen proceso de desarrollo de software
		produce un software de calidad. En este contexto, un plan de calidad es un documento que
		especifica qué procedimientos y recursos deberían aplicarse, quién debe aplicarlos y cuándo
		deberían aplicarse a un proyecto, proceso, producto o contrato específico, de manera de poder
		alcanzar los objetivos de la calidad \parencite{Álvarez_López_2005}, es decir, un documento que
		garantiza la calidad del software elaborado, en base a las expectativas del usuario y a los
		estándares utilizados. El aseguramiento de la calidad (QA) es el proceso encargado de verificar que se estén aplicando
		correctamente los procesos y estándares previamente definidos en el plan de calidad, los cuales
		tiene como objetivo asegurar la calidad del producto final \parencite{Sommerville_2011}.
		
		Este trabajo se realizó en el contexto de la cátedra de Metodología de Sistemas II, con el objetivo
		de aplicar los conocimientos impartidos en clase a través del desarrollo de un plan de calidad
		para el software “Sistema de Gestión Empresarial” (SGE), el cual será aplicado en el software
		antes mencionado para posteriormente ser revisado por un grupo externo.
	\section{Desarrollo de Trabajo}
	Inicio desarrolo del trabajo
	\subsection{Proposito}
	\subsection{Organizacion}
	\subsubsection{Evaluacion de Requerimientos}
	\subsubsection{Evaluación del Diseño del Software}
	\subsubsection{Evaluación del Proceso de Acciones Correctivas}
	\subsubsection{Revisión y Auditoría}
	
	\subsection{Documentación}
	\subsection{Pruebas}
	\subsection{Informe de Problemas y Acción Correctiva}
	\subsection{Gestión de Riesgo}
	\subsubsection{Identificación de Riesgos}
	
	\section{Resultados Obtenidos}
	Inicio resultados obtenidos
	
	\section{Conclusiones}
	Inicio conclusiones.
	
	\nocite{*}
	\section{Referencias}
	\printbibliography[heading=none]
\end{document}
